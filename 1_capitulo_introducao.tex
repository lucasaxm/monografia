\chapter{Introdução}
\label{Introducao}
%internet
O uso da internet no Brasil vem crescendo a cada ano. Dados oficiais do Banco Mundial apontam que em 2013, 51,6\% da população brasileira tinha acesso à internet, colocando o Brasil na 83a posição no ranking mundial de uso de internet\cite{googlePublicData}.

Uma pesquisa publicada em fevereiro de 2014 pela Pew Research Center\cite{pewPage}, indica que 75\% dos adultos do Brasil acessavam a internet diariamente, sendo que destes, 86\% a utilizavam para acessar redes sociais.
    % Colocar um gráfico aqui (Revisão Bruno)
    
O aumento do número de seus usuários, aumentou também o volume de informações geradas.\cite{domoPage} % affonso: achei meio bosta.

% redes sociais
No Brasil, entre as pessoas que acessam a internet, cerca de 73\% usam redes sociais\cite{BrasilRedesSociais}. No mundo cerca de 1.32 bilhões de pessoas são usuários ativos mensais do Facebook, tornando-o a rede social mais acessada do mundo\cite{investorFacebook}. A segunda \textit{mídia social}\footnote{Websites e aplicações que permitem os usuários criar e compartilhar conteúdo ou participar em redes sociais.\cite{socialMedia}}mais acessada é o Youtube com mais de 1 bilhão de usuários\cite{estatisticasYoutube}.

Esse volume de pessoas acessando as redes sociais e o conteúdo gerado através desses domínios, desperta o interesse de várias áreas para essas ferramentas de interação social.

Uma área que possui vários estudos sobre o uso dessa ferramenta de comunicação é a Ciência da Computação\cite{Benevenuto}. Entre estes, existe o estudo sobre extração de dados em redes sociais, que tem como motivação a quantidade de conteúdo gerada através desses sítios. Como exemplo, somente o Youtube gera cerca de 300 horas de vídeos por minuto \cite{estatisticasYoutube}.

Parte das publicações contidas em redes sociais são públicas. É comum encontrar postagens sobre relacionamentos, oportunidades de emprego, opiniões, entre outros. Este tipo de informação pode ser entendido como uma fonte significativa de informação compartilhada de onde é possível fazer, por exemplo, análises de sentimento deste grupo populacional.

Uma das formas de se fazer a análise é através do uso da estatística. Até o começo do século XXI, uma análise estatística era feita através de pesquisas ou entrevistas feitas por voluntários, o que tornava a maioria dos resultados baseados em pequenas amostras de dados, resultando em estatísticas pouco representativas\cite{Benevenuto}.

Através da extração de dados dessas redes, podemos obter informações sobre qualquer assunto em nível mundial, proporcionando uma base de dados alinhada com a realidade.

\section{Objetivo}

O objetivo é descrever um mecanismo de extração da informação contida em redes sociais e submeter esta informação a ferramentas estatísticas para análise. Para tal foi desenvolvida uma aplicação capaz de (1) extrair informações específicas, (2) armazená-la em um banco de dados e (3) aplicar uma análise estatística sobre estas informações.

No capítulo 2 explicaremos as principais ferramentas e conceitos utilizados como: as APIs das redes sociais estudadas, o framework de desenvolvimento, o sistema gerenciador de banco de dados e uma ferramenta estatística.
No capítulo 3 mostraremos uma visão geral da aplicação e seu funcionamento, e no capítulo 4 detalharemos o processo de pesquisa. E por fim apresentaremos os resultados obtidos no capítulo 5, com o auxílio da ferramenta SentiWordNet\footnote{SentiWordNet é um recurso léxico para mineração de opinião que atribui para cada conjunto de palavras sinônimos três "pontuações de sentimento": positividade, negatividade e objetividade \cite{SentiWordNet}.}.

% \section{Organização do trabalho}

% Este trabalho está dividido em quatro partes. A primeira parte trata os conceitos que foram utilizados para o desenvolvimento do tema, como aplicações web escaláveis, programação orientada a eventos, Javascript e NoSQL. 

% Na segunda parte explicamos mais detalhadamente as tecnologias do MEAN que foram propostas, enfatizando algumas características que contribuem para a escalabilidade.

% A terceira parte aborda como os componentes do MEAN são integrados, além de  mostrar um comparativo de uma aplicação MEAN com o LAMP. O LAMP é outra tecnologia bastante difundida, que utiliza o sistema Linux, o servidor Apache, o banco de dados MySQL e a linguagem de programação PHP\footnote{Podendo haver váriações com outras linguagens de programação como Python e Perl}. Os testes de desempenho também serão apresentados nesta parte.

% A conclusão do trabalho e os possíveis trabalhos futuros são apresentados na quarta e última parte.